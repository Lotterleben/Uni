\documentclass[a4paper,10pt]{scrartcl}

% Deutsche Umlaute (Mac)
\usepackage[applemac]{inputenc}

% Deutsche Umlaute (Windows)
%\usepackage[ansinew]{inputenc}

% Einbinden von Bildern
\usepackage{graphicx}

% Seitenfl�che etwas mehr ausnutzen
\usepackage{geometry}
\geometry{a4paper,left=30mm,right=30mm, top=1cm, bottom=3cm}

% Kein Einr�cken bei zweispaltigen Bildunterschriften
\usepackage[normal]{caption}

% Lotte: Glossary benutzen
\usepackage{glossaries}
\loadglsentries[main]{glossary}
\makeglossaries

% gro�er erster Buchstabe
\usepackage{lettrine}

% zeilenumbr�che in tabellen
\newcommand{\specialcell}[2][c]{%
  \begin{tabular}[#1]{@{}c@{}}#2\end{tabular}}

\begin{document}

% Titel
\title{TT1 Praktikumsbericht}
\subtitle{OLSR vs BATMAN}
\author{Alexander Piehl und Lotte Steenbrink}
\date{\today}

% TODO
% fix citations! (-> fix magical plugin)
% spellcheck on

\maketitle

% TODO: add introdtuion subsection?
%Kapitel�bersicht bei Ausarbeitung nicht vergessen!

\section{Vorbereitungen}
Es wurden 2 Router jeweils mit \textbf{OLSR} und \textbf{BATMAN} versehen und waehrend der entsprechenden Durchgaenge im Praktikum jeweils dem passenden Netz hinzugefuegt. Fuer die Durchfuehrung der Messungen wurde ein Rechner mit Mac Os 10.09 mit dem Netz verbunden.\\
Um die Tests und Datenerfassung zu automatisieren, wurde ein kleines python-Skript (TODO: link zum skipt als footnote!) erstellt. Dieses fuehrt zuerst mit jeweils 5 Sekunden Abstand 25 pings durch, wobei fuer jeden Ping der anzupingende Router randomisiert aus einer Liste aller Router ausgewaehlt wird. Die selbe Vorgehensweise wird dann 20-mal mit iperfs an alle verf�gbaren iperf-server wiederholt. 

\section{Durchfuehrung}
Um die generelle Erreichbarkeit sowie packet loss und latenzen aller teilnehmenden Knoten zu testen wurde zunaechst \textbf{ping} verwendet.
Die Verbindung zum allgemeinen iperf-server sowie zu im Netz verteilten iperf-Servern wurde logischerweise mit \textbf{iperf} gemessen.\\
Neben den automatisierten Tests per Skript wurden manuell stichtpobenartig pings und iperf-Anfragen verschickt.\\
Leider haben wir uns bei der Durchfuehrungszeit und -Frequenz verschaetzt, und ping und iperf sequentiell durchgef�hrt, weshalb f�r iperf keine vollstaendigen automatisierten Daten vorliegen. Hier musste auf die Stichprobendaten zurueckgegriffen werden.

% iperfserver immer wieder abgeschmiert -> schwer zuverl�ssig messbar
\section{Beobachtungen}
% leider hat die instabilit�t des netzes und 
Im Folgenden werden zunaechst waehrend der Tests beobachtete Phaenomene beschrieben und dann anhand der erfassten Daten diskutiert.
\subsection{BATMAN}
\begin{itemize}
\item stabile Verbindung zur letzten Reihe und zum Access Point
\item unstabile Verbindungen zu den vorderen Reihen
\item Timeouts beim Pingen, schlechte Metrik
\item generell war die Verbindung stabil, wenn vorhanden
\end{itemize}

\subsection{OLSR}
\begin{itemize}
\item Netz war unstabil
\item Knoten waren mal erreichbar und wieder nicht
\item Access Point war immer wieder �beralstet
\item daher war es schwierig gute Messdaten zu erzeugen
\end{itemize}

\section{Zusammenfassung}
TODO: graphen und beobachtung zusammenfassen

Rueckblickend laesst sich ausserdem sagen, dass es klueger gewesen waere, iperf und ping semi-abwechselnd\footnote{zB 5 pings, ein iperf, da es mehr Knoten als iperf-server gab} durchzuf�hren, um Netzwerkschwankungen besser abzubilden. 
Zusaetzlich haette die Verwendung von \textbf{traceroute} Aufschluss �ber den mittleren und maximalen Hop-Count geben k�nnen.

%{\small
%\bibliographystyle{ieeetr}
%\bibliography{kurzausarbeitung}
%}


\end{document}