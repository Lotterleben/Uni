\documentclass{acm_proc_article-sp}

% Glossary benutzen
\usepackage{glossaries}
\loadglsentries[main]{glossary}
\makeglossaries
%\printglossaries

\begin{document}

\title{Routing protocol evaluation for the IoT}
\subtitle{Requirement analysis and experiment design for large-scale test beds.}

\author{
\alignauthor
Lotte Steenbrink
       \email{lotte.steenbrink@haw-hamburg.de}
}

\maketitle
\begin{abstract}
TODO
\end{abstract}

\keywords{IoT, routing, MANET, test beds}

\section{Introduction}
\label{sec:Intro}
%===============================================================================
The \gls{IoT} is both a growing market and a budding research field. One central aspect of IoT communications is routing: finding the best paths between nodes and towards sink nodes and gateways is crucial to ensure energy-efficient and smooth network operations. However, practical experience with IoT routing is sparse, and scientific evaluation of such environments is rare. Most routing protocol evaluations are simulation-based, and even fewer of these evaluations have been designed with the IoT in mind.
This paper presents a testbed-based evaluation approach tailored to the IoT. The goal is to enable the evaluation of routing protocols which have been created for \glspl{LLN} or \glspl{MANET} with regard to their suitability for the IoT.

\subsection{Related work}
\label{subsec:related_work}
%...............................................................................
While testbed experiments are rare, research on the foundations needed to conduct life-like experiments has been done for about two decades, and is increasingly focused on the IoT.\\
\cite{RFC-2501} provides a summary of issues which should be considered when evaluating a routing protocol. Routing requirements for the IoT-like scenarios of home and building automation, as well as urban \glspl{LLN} are described in \cite{RFC-5826},  \cite{RFC-5867} and \cite{RFC-5548}.
\cite{food_monitoring} discusses influences on transmission range in food monitoring use cases, in particular monitoring bananas during transport. results were achieved both through mathematical analysis as well as well as a simple testbed consisting of four nodes.
\cite{testbed-survey} presents the features and failings of different Wireless Sensor Network Testbeds, along with a requirement analysis for IoT-ready testbeds.

%Suchworte: Experiment Design

\section{IoT Scenarios}
\label{sec:Scenarios}
%===============================================================================
%What do IoT-like topologies, environments, scenarios look like? Do they maybe need to be divided into different categories?
%-> Short overview, then pick a strong one (use case exists in practice, is not niche etc) as a first step. Go on to model only that one.

In order to be able to create an accurate model, the core characteristics which make up the experiment scenario have to be determined. These characteristics are: Network topology, traffic patterns, mobility patterns (if any), energy efficiency requirements and radio propagation characteristics and the environmental factors which may influence them.
%TODO: elaborate if I need more text
Since the IoT is a paradigm which encompasses many different use cases and environments, there is no such thing as \emph{the} typical IoT scenario. A building automation installation in a factory might feature a star topology with TODO traffic, no mobility, low energy efficiency requirements and an open field, resulting in a wide radio range (TODO: Quelle), while a solution monitoring the insides of a food truck features a mesh topology made necessary by the high density of the truck's contents which result in low radio ranges, and bursty traffic and node mobility whenever the goods are unloaded or rearranged \cite{food_monitoring}.

TODO: list taxonomy of scenarios (-> see BA notizen)

Therefore, providing a ``one scenario fits all'' solution is out of scope for this paper. Instead, a specific scenario will be studied and modeled in detail, with the hope that some of the building blocks may be reused as research expands.
To achieve this, TODO has been chosen as the scenario to be modeled, as it can be found in a wide range of applications, and its characteristics are the most challenging for routing protocols.

\section{Experiment goals}
\label{sec:Goals}
%===============================================================================
What do I actually want to investigate? Which routing protocol eigenschaften do I want to check, what do I expect from a routing protocol under IoT conditions in terms of performance, reliability etc?
Use \cite{RFC-2501} as a ref! And maybe my notes from back then?

\section{Experiment design}
\label{sec:Design}
%===============================================================================
Based on the goals: Which topology/topologies, which network size(s), which use cases, how do I want to model them, in how much detail.. etc -> Welche Aspekte der Realität sollen abgebildet werden?\\
Was und wie wird ausgewertet?

\subsection{Choosing the testbed}
\label{subsec:testbed_choice}
%...............................................................................
Goooo IoT Lab! (see AW2 paper here)
Hat das IoT-Lab irgendwelche fancy experimentierhilfen wie nepi sie zur verfügung stellt?

\subsection{Experiment Setup}
\label{subsec:setup}
%...............................................................................
How many nodes, which communication patterns, which mobility patterns (if any), which arrangement...

\subsection{Experiment evaluation}
\label{subsec:evaluation}
%...............................................................................
Which data do I want to collect and evaluate? What do I want to look for? Is there anything I want to show?


\section{Conclusion and outlook}
\label{sec:Conclusion}
%===============================================================================

Outlook: Actually implement this. (Say with which RPs!)

\bibliographystyle{ieeetr}
\bibliography{sigproc}
\end{document}
