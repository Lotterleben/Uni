\documentclass[a4paper,10pt]{scrartcl}

% Deutsche Umlaute (Mac)
\usepackage[applemac]{inputenc}

% Deutsche Umlaute (Windows)
%\usepackage[ansinew]{inputenc}

% Einbinden von Bildern
\usepackage{graphicx}
\usepackage{todonotes}

% SeitenflŠche etwas mehr ausnutzen
\usepackage{geometry}
\geometry{a4paper,left=30mm,right=30mm, top=1cm, bottom=3cm}

% Kein EinrŸcken bei zweispaltigen Bildunterschriften
\usepackage[normal]{caption}

% == Lottes Änderungen ===
% großer erster Buchstabe
\usepackage{lettrine}

% Glossary benutzen
\usepackage{glossaries}
\loadglsentries[main]{glossary}
\makeglossaries

% zeilenumbrüche in tabellen
\newcommand{\specialcell}[2][c]{%
  \begin{tabular}[#1]{@{}c@{}}#2\end{tabular}}

\begin{document}

% Titel
\title{Hybrid routing for the Internet of Things}
\subtitle{challenges and opportunities}
\author{Lotte Steenbrink}
\date{Wintersemester 2014/15}
\maketitle

\begin{abstract}


\end{abstract}

\section{Introduction}
\label{sec:Intro}
%==============================================================================

\subsection{What is the Internet of Things?}
\label{subsec:IoT}
%==============================================================================
The \gls{IoT} envisions autonomous communication between tiny computers installed in everyday objects such as furniture, toys, clothing, or tools with the goal of making them smarter and improving their user experience. IoT devices typically are very constrained devices with no constant power supply. Therefor, they need to be resourceful in terms of computation, storage, RAM and energy usage.
To communicate amongst each other, IoT nodes form spontaneous, wireless mesh networks.
The vision of the Internet of Things is rapidly becoming reality, and with its rise, new demands for routig protocols that serve these kinds of mesh networks arise.


\subsection{What is Hybrid Routing?}
\label{subsec:hybrid}
%==============================================================================


\section{related work}
\label{sec:related_work}
%==============================================================================
Talk about the past, mostly. briefly mention experimental work, esp emmanuels, reference 

\section{Key aspects of Hybrid Routing Protocols}
\label{sec:key_aspects}
%==============================================================================
The protocols discussed in \ref{sec:related_work} share common approaches, some of which fundamentally shape the way a routing protocols sees and serves a network. The following sections will identify these key aspects and discuss them with regard to the requirements of IoT environment


\subsection{Scope}
\label{subsec:scope}
%==============================================================================

\subsection{Architecture}
\label{subsec:architecture}
%==============================================================================

\section{Suitability for the IoT}
\label{sec:key_aspects}
%==============================================================================

\section{Experimental work}
\label{sec:experiments}
%==============================================================================


\section{Conclusion}
\label{sec:conclusion}
%==============================================================================
Framework good, protocol bad. scope depends on deployment. framework may be able to accomodate that too. simulations mostly suck. real-world experiences are rare. hooray for testbeds!

\section{Outlook}
\label{sec:outlook}
%==============================================================================
figure out ways to slim down frameworks.
get real-world experiences. verify vermutungen oder auch nicht. so much to do!

\printglossaries

{\small
\bibliographystyle{ieeetr}
\bibliography{ausarbeitung}
}


\end{document}