\documentclass[a4paper,10pt, twocolumn]{scrartcl}

% Deutsche Umlaute (Mac)
\usepackage[applemac]{inputenc}

% Deutsche Umlaute (Windows)
%\usepackage[ansinew]{inputenc}

% Einbinden von Bildern
\usepackage{graphicx}

% Seitenfl�che etwas mehr ausnutzen
\usepackage{geometry}
\geometry{a4paper,left=30mm,right=30mm, top=1cm, bottom=3cm}

% Kein Einr�cken bei zweispaltigen Bildunterschriften
\usepackage[normal]{caption}

% Lotte: Glossary benutzen
\usepackage{glossaries}
\loadglsentries[main]{glossary}
\makeglossaries

\begin{document}

% Titel
\title{Hybrid routing for the IoT}
\subtitle{challenges and opportunities}
\author{Lotte Steenbrink}
\date{Wintersemester 2014/15}

\maketitle

\begin{abstract}

\end{abstract}

\section{Introduction}
% motivation: Why do we need hybrid routing protocols?
% -> concrete use cases

%Kapitel�bersicht bei Ausarbeitung nicht vergessen!
Routing in

\section{Challenges}

\section{Existing approaches}
% ZRP
Existing approaches can be categorized as either area-centered or route-centered.\\
\emph{Area-centered} routing protocols, such as the \gls{ZRP}\cite{ZRP-Draft} or (TODO: more protocols+ sources),
distinguish routes by locality: nearby routes are maintained in a proactive fashion, while routes to destinations farther away in the network are maintained reactively.\\
\emph{Route-centered} protocols like the P2P extension\cite{RFC-6997} of the \gls{RPL}\cite{RFC-6550} instead focus on the importance of a route: a connection to a \gls{sink node} may be crucial to the operation of a network and should be maintained proactively, while connections between arbitrary nodes
may be short-lived and should only be established on-demand (i.e. reactively) in order to save resources and bandwidth.
\section{Conclusion}

% Alternativ kann auch BibTex verwendet werden
%\begin{thebibliography}{}
%\bibitem[Jenke, 2014]{jenke14} P. Jenke und S. Sarstedt: {\em AW2 ist ein wichtiger Baustein auf dem Weg zum Master}, Journal of NichtVer�ffentlicht, 2014
%\bibitem[Wolf]{wolf} Thomas Wolf: {\em www.foto-tw.de}.
%\end{thebibliography}

{\small
\bibliographystyle{ieeetr}
\bibliography{kurzausarbeitung}
}


\end{document}