\section{Conclusion and outlook}
\label{sec:conclusion}
%==============================================================================
A wide range of approaches for routing in the \gls{IoT} have been presented. Candidate protocols employing the approaches suggested have been introduced, along with possible criteria a routing protocol may have to match in order to be suitable for the IoT. While none of the protocols may be a one size fits all-solution, they may be suitable for specific IoT scenarios.\\
However, most proposed solutions have never been tested or even simulated in deployments which match the requirements listed in \ref{subsec:intro_requirements}.
It has been argued that there is a need for testing procedures tailored to IoT environments. The creation and adaption of standardized benchmarks for routing in the Internet of Things may advance the comparison of candidate protocols for the IoT.\\
Furthermore, the topology and attributes of the IoT complicate experiment and simulation setup: while the former is expensive to set up and maintain, the latter quickly fails to represent the network properties and influences correctly. \cite{iot_experiment_facilities} provides an overview of publicly accessible testbeds suitable for the IoT which may be used for future research and discusses their challenges and application areas.\\
Wrapping up, it can be concluded that there already are many existing approaches which may prove to be suitable for the IoT. Their direct comparison in both simulation and realistic testbed scenarios could provide further insight into their suitability for distinct IoT scenarios and reveal optimization potential.